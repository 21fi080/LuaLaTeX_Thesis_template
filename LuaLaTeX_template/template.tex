% !TEX program = lualatex
\documentclass[a4paper,10pt,twocolumn]{ltjsarticle}
\usepackage{amsmath}
\usepackage{amssymb}
\usepackage{graphicx}
\usepackage{booktabs}
\usepackage{url}
\usepackage{cite}
\usepackage{indentfirst}

% --- 1. 用紙・余白設定 ---
\usepackage[top=20mm, bottom=30mm, left=20mm, right=20mm]{geometry}

% --- 2. フォント設定 (LuaLaTeX) ---
\usepackage{luatexja-fontspec}

% 欧文フォント: Times New Roman
\setmainfont{Times New Roman}[Ligatures=TeX]

% 日本語フォント: MS明朝, 太字はMSゴシック
\setmainjfont{MS Mincho}[
  BoldFont = MS Gothic,
  YokoFeatures = {JFM=prop}
]
% \sffamily (サンセリフ) を呼び出した時はMSゴシック
\setsansjfont{MS Gothic}

% 簡体字フォント設定 (scale=0.93)
\newjfontfamily\chnfont{FandolSong-Regular}[
  BoldFont = FandolHei-Regular,
  Scale = 0.93
]
\newcommand{\chn}[1]{{\chnfont #1}}

% --- 3. 見出し・キャプション設定 ---
\usepackage{titlesec}
\usepackage{caption}

% 章題 (Section): MSゴシック, 12pt
\titleformat{\section}
  {\large\bfseries\sffamily}
  {\thesection}{1em}{}

% 節題 (Subsection): MSゴシック, 10pt
\titleformat{\subsection}
  {\normalsize\bfseries\sffamily}
  {\thesubsection}{1em}{}

% キャプション: MS明朝, 10pt
\captionsetup{font={rm,md}}
\renewcommand{\figurename}{図}
\renewcommand{\tablename}{表}

% ページ番号を削除
\pagestyle{empty}

% --- 文書開始 ---
\begin{document}

% --- タイトル部 ---
\twocolumn[
    \begin{center}
        % 研究表題(日本語): MSゴシック, 14pt
        {\fontsize{14pt}{20pt}\selectfont \sffamily \bfseries
            日本語と簡体字の混植に対応した卒論梗概テンプレート
        } \\
        \vspace{5mm}

        % 研究表題(英語): Times New Roman (Bold), 14pt
        {\fontsize{14pt}{20pt}\selectfont \bfseries
            Thesis Template Supporting Mixed Japanese and Simplified Chinese Text
        } \\
        \vspace{5mm}

        % 著者情報: MS明朝/Times, 12pt
        % ここを表組み(tabular)に変更して中央揃えを実現
        {\large
            \begin{tabular}{c@{\hskip 3em}c} % 学生と教員の間に少し広めの空白(3em)を設定
                99FI999 電大 未来子 & 指導教員 未来 太郎 \\
                Mikiko Dendai  & Taro Mirai
            \end{tabular}
        }
    \end{center}
    \vspace{5mm}
]

% --- 本文開始 ---

\section{はじめに}
本稿では,Lua\LaTeX を用いた多言語混在文書の作成手法について述べる.
これは本文の日本語(MS明朝 10pt)のテストである.
句読点は全角の「,」「.」を用いる.

英語のテスト.This is the main text in Times New Roman (10pt).

\section{簡体字の表示について}
従来のp\LaTeX 環境では困難であった簡体字の直接記述を,Lua\TeX -jaの機能を用いて実現した.

\begin{itemize}
    \item 日本語の骨:骨(MS明朝)
    \item 簡体字の骨:\chn{骨}(FandolSong, Scale 0.93)
\end{itemize}

文章中の混在テスト:
\chn{中国语的测试。这里使用了 FandolSong 字体,并且调整了大小以匹配日文字体。}
このように,違和感なく混植が可能である.

\section{体裁の確認}
\subsection{フォント設定}
表\ref{tab:fonts}に,指定されたフォント設定を示す.
章題はMSゴシック12pt,節題はMSゴシック10ptで設定されている.

\begin{table}[h]
    \centering
    \caption{フォント設定の確認}
    \label{tab:fonts}
    \begin{tabular}{lll}
        \toprule
        項目  & 和文フォント             & 欧文フォント                          \\
        \midrule
        表題  & {\sffamily MSゴシック} & \textbf{Times New Roman (Bold)} \\
        本文  & {\rmfamily MS明朝}   & Times New Roman                 \\
        簡体字 & \chn{宋体 (SongTi)}  & -                               \\
        \bottomrule
    \end{tabular}
\end{table}

\subsection{図表の配置}
図表のキャプションはMS明朝10ptとし,1行の場合は中央揃えとなる設定を適用している.

\section{まとめ}
本テンプレートはあくまで非公式であることに注意し,使用に際しては自己責任で行うこと.
VScode以外のエディタでは,動作確認を行っていないため,適宜調整が必要となる可能性がある.
また,最低限のパッケージのみを導入しているため,必要に応じて追加のパッケージを導入することが望ましい.

% 参考文献
\begin{thebibliography}{9}
    \bibitem{ref1} 著者名, ``表題,'' 雑誌名, vol.1, no.1, pp.1-8, 2024.
\end{thebibliography}

\end{document}